%%%%%%%%%%%%%%%%%%%%%%% file template.tex %%%%%%%%%%%%%%%%%%%%%%%%%
%
% This is a general template file for the LaTeX package SVJour3
% for Springer journals.          Springer Heidelberg 2010/09/16
%
% Copy it to a new file with a new name and use it as the basis
% for your article. Delete % signs as needed.
%
% This template includes a few options for different layouts and
% content for various journals. Please consult a previous issue of
% your journal as needed.
%
%%%%%%%%%%%%%%%%%%%%%%%%%%%%%%%%%%%%%%%%%%%%%%%%%%%%%%%%%%%%%%%%%%%
%
% First comes an example EPS file -- just ignore it and
% proceed on the \documentclass line
% your LaTeX will extract the file if required
\begin{filecontents*}{example.eps}
%!PS-Adobe-3.0 EPSF-3.0
%%BoundingBox: 19 19 221 221
%%CreationDate: Mon Sep 29 1997
%%Creator: programmed by hand (JK)
%%EndComments
gsave
newpath
  20 20 moveto
  20 220 lineto
  220 220 lineto
  220 20 lineto
closepath
2 setlinewidth
gsave
  .4 setgray fill
grestore
stroke
grestore
\end{filecontents*}
%
\RequirePackage{fix-cm}
%
%\documentclass{svjour3}                     % onecolumn (standard format)
%\documentclass[smallcondensed]{svjour3}     % onecolumn (ditto)
\documentclass[smallextended]{svjour3}       % onecolumn (second format)
%\documentclass[twocolumn]{svjour3}          % twocolumn
%
\smartqed  % flush right qed marks, e.g. at end of proof
%
\usepackage{graphicx}
%
% \usepackage{mathptmx}      % use Times fonts if available on your TeX system
%
% insert here the call for the packages your document requires
%\usepackage{latexsym}
% etc.
%
% please place your own definitions here and don't use \def but
% \newcommand{}{}
%
% Insert the name of "your journal" with
 \journalname{Journal of Homotopy and Related Structures}
%
\begin{document}

\title{Grothendieck Topologies for the Simplex Category}


%\titlerunning{Short form of title}        % if too long for running head

\author{Frank Murphy-Hernandez}

%\authorrunning{Short form of author list} % if too long for running head

\institute{Frank Murphy-Hernandez \at
              Facultad de Ciencias, Universidad Nacional Aut\'onoma de M\'exico, Ciudad Universitaria\\ Ciudad de M\'exico, M\'exico\\
              \email{murphy@ciencias.unam.mx} 
}

\date{Received: date / Accepted: date}
% The correct dates will be entered by the editor


\maketitle

\begin{abstract}
Toda subcategorías reflectiva de una categoría de pregavillas es una categoría de gavillas.

En conjuntos simpliciales los complejos de Kan, son una subcategorías reflectiva???
\keywords{Simplicial Set \and Sheaf \and Grothendieck Topology}
% \PACS{PACS code1 \and PACS code2 \and more}
% \subclass{MSC code1 \and MSC code2 \and more}
\end{abstract}

%%%%%%%%%%%%%%%%%%%%%%%%%%%%%%%%%%%%%%%%%%%%%%%%%%%%%%%%%%%%%%%%%%%%%%%%%%%%%
\section{Introduction}
\label{intro}
Your text comes here. Separate text sections with


%%%%%%%%%%%%%%%%%%%%%%%%%%%%%%%%%%%%%%%%%%%%%%%%%%%%%%%%%%%%%%%%%%%%%%%%%%%%%
\section{Preliminaries}
\label{sec:1}

We denote the simplex category as $\Delta$. It is the category of non-empty finite ordinals and monotone maps. For a natural number $n$, we put $[n]$ as $\{0,\dots n\}$. In this manner $[n]$ is the ordinal $n+1$.

We denote the category of sets and functions as $\mathcal{S}$. A simplicial set $K$ is contravariant functor from the simplex category $\Delta$ into $\mathcal{S}$. For reference of simplicial sets, we recommend \cite{goerss2009simplicial} and \cite{may1992simplicial}.

For a category $\mathcal{A}$ and an object $A$ in $\mathcal{A}$, a sieve $S$ over $A$ is a family of morphisms in $\mathcal{A}$ with codomain $A$ such that $fg\in S$, if $f\in S$ and $fg$ is defined. We denote the family of sieves over $A$ as $\Omega(A)$. We have that $\Omega(A)$ is ordered by the inclusion. Moreover, the intersection of sieves is a sieve. So for a family of morphisms $\mathcal{X}$ with codomain $A$, we have the sieve $c(\mathcal{X})$ generated by $\mathcal{X}$. In particular, for a morphism $f$ in $\mathcal{A}$, we have that $c(f)=\{fg\in\mathcal{A}\mid cod(g)=dom(f)\}$. We observe that $c(\mathcal{X})=\bigcup_{f\in\mathcal{X}}c(f)$. Thus for  $\mathcal{X}$ and $\mathcal{Y}$ families of morphisms with codomain $A$, if $\mathcal{X}\subseteq \mathcal{Y}$ then $c(\mathcal{X})\subseteq c(\mathcal{Y})$.

The references that we recommend for topos theory are \cite{johnstone2014topos} and \cite{maclane2012sheaves}.

%%%%%%%%%%%%%%%%%%%%%%%%%%%%%%%%%%%%%%%%%%%%%%%%%%%%%%%%%%%%%%%%%%%%%%%%%%%%%
\section{Sieves in the simplex category}
\label{sec:2}

\begin{proposition}
Let $f\colon[m]\longrightarrow [n]$ be a morphism in $\Delta$. Then 
\[c(f)=\{g\colon [k]\longrightarrow [n]\in\Delta\mid im(g)\subseteq im(f)\}
\]
\end{proposition}

\begin{proof}
Let $g\in c(f)$. Then there is $h\in\mathcal{A}$ such that $g=fh$. It follows that $im(g)\subseteq im(f)$.

Let $g\colon [k]\longrightarrow[n]$ with $im(g)\subseteq im(f)$. We build a function $h\colon [k]\longrightarrow [m]$ given by $h(x)=\min\{y\in [n]\mid f(y)=g(x)\}$. As $im(g)\subseteq im(f)$, the set $\{y\in [n]\mid f(y)=g(x)\}$ is not empty for any $x\in [k]$. By construction $h$ satifies that $g=fh$. We define $A_x=\{y\in [n]\mid g(x)\leq f(y)\}$ for $x\in [k]$. So $h(x)=\min A_x$. If $x\leq x'$ in $[k]$, then $A_{x'}\subseteq A_x$. Thus $h(x)=\min A_x\leq A_{x'}=h(x')$. Therefore $h$ is monotone and $g=fh$.
\end{proof}

\begin{proposition}
Let $S$ be a sieve over $[n]\in\Delta$. Then there is a minimal family $\mathcal{X}$ contained in $S$ such $c(\mathcal{X})=S$. Moreover, $\mathcal{X}$ is finite.
\end{proposition}

\begin{proof}

\end{proof}

\begin{proposition}
Let $f\colon [m]\longrightarrow [n]$ and $\alpha\colon [k]\longrightarrow [n]$ be two morphisms in $\Delta$. Then
\[
\alpha^*(c(f))=\{g\colon [l]\longrightarrow[m]\in\Delta\mid im(g)\subseteq \alpha^{-1}(im(f))\}.
\]
\end{proposition}

\begin{proof}

\end{proof}

%%%%%%%%%%%%%%%%%%%%%%%%%%%%%%%%%%%%%%%%%%%%%%%%%%%%%%%%%%%%%%%%%%%%%%%%%%%%%
\section{Grothendieck Topologies in the simplex category}
\label{sec:3}

%%%%%%%%%%%%%%%%%%%%%%%%%%%%%%%%%%%%%%%%%%%%%%%%%%%%%%%%%%%%%%%%%%%%%%%%%%%%%
\section{Simplicial Sets as Sheaves}
\label{sec:4}


%%%%%%%%%%%%%%%%%%%%%%%%%%%%%%%%%%%%%%%%%%%%%%%%%%%%%%%%%%%%%%%%%%%%%%%%%%%%%

\bibliography{biblio}
\bibliographystyle{plain}



\end{document}


